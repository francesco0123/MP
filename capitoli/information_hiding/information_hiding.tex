\chapter{INFORMATION HIDING}

Utilizzo di meccanismi per evitare che tutti possano accedere a certe parti di codice.\\
Noi abbiamo visto i livelli di accessibilità in Java

\begin{figure}[H]
  \centering
  \includegraphics[width=5cm]{information_hiding/accessibilità}
\end{figure}

Per convenzione

\begin{itemize}
  \item le variabili d'instanza dovrebbero essere private (evita anche protected);
  \item i campi static, per rappresentare costanti accessibili a tutti, dovrebbero essere public final;
  \item i metodi che utilizzano dettagli interni dovrebbero essere private;
  \item i metodi che potrebbero servire alle sottoclassi, dovrebbero essere protected;
  \item i metodi che saranno usati dalle altre classi, dovranno essere public.
\end{itemize}

Creare un oggetto che ha parametri d'instanza, privati, che però vi si possonono accedere, con un get, o settare, tramite un setter, è la cosa sbagliata.

Supponiamo di avere una classe Person tale che

\begin{lstlisting}
  public class Person {
    private String name;
    
    public Person(String name) {
      this.name = name;
    }
    
    @Override
    public String toString() {
      return "Person [name=" + name + "]";
    }
}
\end{lstlisting}

Se instaziassi la classe e cercassi di accedere al campo name, il compilatore mi segnalerebbe errore.

L'information hiding è pensato per essere usato a tempo di compilazione e non a run time, java permette di accedere alle variabili, anche private, di una classe, 
infatti

\begin{lstlisting}
  @Test public void testReadPrivateMember() throws ... {
    Person person = new Person("John");
    // riferimento al campo (anche se privato)
    Field nameField = person.getClass().getDeclaredField("name");
    // aggira il controllo di accesso
    nameField.setAccessible(true);
    // legge il valore del campo
    Object nameValue = nameField.get(person);
    assertEquals("John", nameValue);
  }
\end{lstlisting}

\textbf{N.B.} Il metodo setAccessible(boolean bol) permette, effettivamente di accedere ai campi private, altrimenti non sarebbe possibile farlo.
\medskip

Se io dichiarassi name public, legittimerei il suo uso da parte dei cliente, imponendomi il vincoli non poter mai più modificare quella variabile, altrimenti i client 
verranno rotti (non compileranno più).

Ogni client che userà quel membro sarà strettamente accoppiato (tightly coupled) a quel membro (noi questo non lo vogliamo), stessa cosa per protected, solo che qui 
saranno le sottoclassi ad essere tightly coupled.

Dichiarando un membro private, lo nascondo al "mondo", non legittimiamo il suo uso da parte dei client, il compilatore controllerà il corretto uso nei vari client e
\begin{itemize}
  \item se compileranno con errori di accesso, significa che i client non sono legittimati all’uso;
  \item se accederanno al membro via reflection (run time) e qualcosa non dovesse funziona, sarà colpa loro. 
\end{itemize} 

Più dettagli interni nascondiamo, più avremo disaccoppiamento e più sarà facile testare singole componenti in isolamento.

Se dichiarassimo qualcosa come public/protected, come le API, e volessimo effettuare dei cambiamento, dovremmo notificare i client quando rilasceremo il nuovo 
aggiornamento, specificando se si tratta di qualcosa che rompe API, implementa qualcosa di nuovo oppure la risoluzione di bug.

Un modo per indicare questi cambiamenti è l'utilizzo della semantic versioning, un numero composta da tre parti, $X.Y.Z$, dove se
\begin{itemize}
  \item dovesse cambiare X, significherebbe aver introdotto qualcosa che rome l'API;
  \item dovesse cambiare Y, significherebbe aver introdotto qualcosa di nuovo;
  \item dovesse cambiare Z, significherebbe aver risolto dei bug (sperando di non aver introdotto nuovi bug).
\end{itemize}