\chapter{COVARIANZA E CONTROVARIANZA}

Dalla matematica, una funzione \textit{f}, tale che f$:D\rightarrow C$, può essere sosituita da una funzione \textit{g}, tale che g$:D^\prime\rightarrow C^\prime$, sse $C^\prime<:C$ 
(\textbf{covariante} sul tipo di \textbf{ritorno}) e $D<:D^\prime$ (\textbf{controvariante} su tipo del \textbf{parametro}), ovvero una funzione g, che deve 
sostituire f, può essere più specifica sul tipo di ritorno e più generale sul parametro in input.

Supponiamo di avere due classi, Employee e Manager, tale che Manager$<:$Employee.

Employee ha un metodo setSalary() e Manager ha il metodo setBonus(int).

Prendiamo altre due classi, EmployyeDB e ManagerDB, tale che ManagerDB$<:$EmployyeDB.

La prima ha al suo interno un metodo find(int) che restiusce un Employee, mentre il secondo ridefinisce il metodo facendo tornare un Manager.
\begin{multicols}{2}
    \begin{lstlisting}
    public class EmployeeDatabase {
        public Employee findEmployee(int id) {
            ...
        }
    }    
    \end{lstlisting}       
\columnbreak
    \begin{lstlisting}
    public class ManagerDatabase extends EmployeeDatabase {
        @Override
        public Manager findEmployee(int id) {
            ...
        }
    }
    \end{lstlisting}
\end{multicols}

Se eseguissi questo codice, tutto funzionerebbe.

\begin{lstlisting}
    EmployeeDB d = new ManagerDB();
    Employee e = d.find(1);
\end{lstlisting}

Per il binding dinamico, d staticamente è EmployeeDB ma a runtime è ManagerDB, inoltre il metodo find resituisce un Manager e non abbiamo problemi anche perchè è un 
sottotipo di Employee.

Questo è consistente col tipo di ritorno che è covariante e che può essere uguale o più specializzato.

Ipotizziamo che il concetto di covarianza valga anche per i parametri e quindi supponiamo che EmployeeDB abbia un metodo updateEmployee(Employee), che setta il salario di un Employee, e che la classe sottotipo, ridefinisca il metodo cambiandone il parametro in un Manager,  e aggiungendo al suo interno il metodo setBonus(int) (può farlo in quanto è di Manager).

\begin{multicols}{2}
    \begin{lstlisting}
        public class EmployeeDB {
        ...
            public void updateEmployee(Employee e) {
                e.setSalary(...);
            }
        }
    \end{lstlisting}    
    \columnbreak
    \begin{lstlisting}
        public class ManagerDB extends EmployeeDB {
        ...
            @Override
            public void updateEmployee(Manager e) {
                super.updateEmployee(e);
                e.setBonus(0);
            }
        }
    \end{lstlisting}
\end{multicols}

Eseguendo questo codice

\begin{lstlisting}
    EmployeeDatabase d = new ManagerDatabase();
    d.updateEmployee(new Employee());
\end{lstlisting}

per il binding dinamico, d staticamente è EmployeeDB ma a runtime è ManagerDB, quindi il metodo che sarà chiamato, sarà quello di ManagerDB che 
però non esiste e quindi lancerà un'eccezione del tipo NoSuchMethodException.

A livello di compilazione, tutto è ok perchè il metodo updateEmployee(Employee), staticamente è di tipo \newline EmployeeDB ma a runtime chiamerà il metodo di ManagerDB 
dove abbiamo deciso di mettere in posizione covariante anche i parametri e questo non va bene perchè, per il binding dinamico, verrebbe chiamato updateEmployee
di ManagerDatabase che peraltro si aspettava un Manager ed invece si ritrova un Employee e boom, eccezione del tipo NoSuchMethodException.

Java adotta un sistema di tipi \textbf{sound}, ovvero un programma, accettato staticamente dal compilatore, non generarà mai errori NoSuchMethodException a runtime.

Nello specifico i tipi statici, controllati durante la compilazione, durante l’esecuzione si mantengono come tali oppure diventano sottotipi.

Java non permette la controvarianza durante la ridefinizione di un metodo in quanto richiede che i tipi dei parametri non cambino affatto 
(si dice che java è invariante sul tipo dei parametri).

Supponiamo che Employee abbia un supertipo, Person e che EmployeeDB sia esteso da un'altra classe, PersonDB che esegue override del metodo chiamando dentro il suo 
metodo setName().

\begin{lstlisting}
public class EmployeeDatabase {
    ...
    public void updateEmployee(Employee e) {
        e.setSalary(...);
    }
}

public class PersonDataBase extends EmployeeDatabase {
    @Override
    public void updateEmployee(Person e) {
        super.updateEmployee(e);
        e.setName("");
    }
}
\end{lstlisting}

Dal punto di vista della controvarianza, ciò è corretto, però avremo un errore, in quanto non potremmo usare il metodo super perchè richiede Employee, mentre gli stiamo 
passando un Person e un Person NON è un Employee.

Quindi possiamo dire che Java, per la ridefinizione di metodi, adotta l'approcio sicuro, ovvero covariante sul tipo di ritorno e invariante sui parametri.

Magari una soluzione potrebbe essere quella di preferire l'overloading all'override. 

Magari abbiamo lo stesso metodo, uno che prende in input Employee e l'altro che prende
Person.

\begin{lstlisting}
public class EmployeeDatabase {
    ...
    public void updateEmployee(Employee e) {
        e.setSalary(...);
    }
}

public class ManagerDatabase extends EmployeeDatabase {
    ...
    public void updateEmployee(Manager e) {
        super.updateEmployee(e);
        e.setBonus(0);
    }
}
\end{lstlisting}

Anche qui ci saranno problemi, infatti dato il seguente spezzone di codice

\begin{lstlisting}
EmployeeDatabase d1 = new ManagerDatabase();
ManagerDatabase d2 = new ManagerDatabase();

// chiama EmployeeDatabase.updateEmployee(Employee)
d1.updateEmployee(new Employee());

// chiama ManagerDatabase.updateEmployee(Manager)
d2.updateEmployee(new Manager());

/*  chiama ANCORA EmployeeDatabase.updateEmployee(Employee) perche' d1 e' staticamente EmployeeDatabase
 *   ManagerDatabase.updateEmployee(Manager) aggiunge un metodo in overloading ma NON ridefinisce 
 *   EmployeeDatabase.updateEmployee(Employee) quindi non si applica il binding dinamico!
*/
d1.updateEmployee(new Manager());
\end{lstlisting}

Quindi potremmo risolvere passando dall'usare l'overloading in una gerarchia di classi, all'usare l'overloading in una singola classe.

\begin{lstlisting}
public class EmployeeDatabase {
    ...
    public void updateEmployee(Employee e) {
        e.setSalary(0);
    }

    public void updateEmployee(Manager e) {
        e.setBonus(0);
    }
}
\end{lstlisting}

però anche nel prossimo spezzone, avremo un comportamento non prevedibile
\begin{lstlisting}
EmployeeDatabase d1 = new ManagerDatabase();

// chiama updateEmployee(Employee): l'argomento e' un Employee
d1.updateEmployee(new Employee());

// chiama updateEmployee(Manager): l'argomento e' un Manager
d1.updateEmployee(new Manager());

Employee e1 = new Employee();
Employee e2 = new Manager();

// chiama updateEmployee(Employee)
d1.updateEmployee(e1);

// chiama sempre updateEmployee(Employee)
d1.updateEmployee(e2);
\end{lstlisting}

perchè in java, l'overloading è un meccanismo statico.

Implementare una sorta di overloading dinamico è difficile dal punto di vista dell'efficienza (diverso dalla complessità costante del binding dinamico).

Quindi java, per la ridefinizione dei metodi, adotta l’approccio sicuro, covariante sul tipo di ritorno e invariante sui tipi dei parametri.

Per le lambda invece si, possiamo essere controvarianti sui parametri, esempio il metodo map degli stream che è definito come 
\begin{lstlisting}
<R> Stream<R> map(Function<? super T, ? extends R> mapper)    
\end{lstlisting}

ovvero, una lambda di tipo Function$<$T, R$>$ può accettare anche una lambda di (sotto)tipo Function$<$T’, R’$>$ dove T $<:$ T’ e R’ $<:$ R