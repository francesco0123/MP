\chapter{Visitor}

E' un pattern \underline{comportamentale}, descrive come classi e oggetti interagiscono e si distribuiscono le responsabilità. 
\smallskip

Lo si utilizza quando si ha una struttura di oggetti, appartenenti a classi differenti, avente un supertitpo in comune su cui vogliamo eseguire operazioni sugli 
oggetto della struttura in base alla propria classe concreta senza "inquinare" le classi esistenti.

\section{Funzionamento}

Avremo un'interfaccia comune, AbstractVisitor, che definisce metodi del tipo visitXXX per ogni tipo \underline{concreto} della struttura e le sottoclassi concrete 
che estendono AbstractVisitor.

Nella classe base/interfaccia della gerarchia aggiungiamo il metodo accept(...) che prende in input un AbstractVisitor e le classi concrete della gerarchia 
implementeranno tale metodo chiamando il corrispettivo metodo visitXXX passando se stesso come input.
\medskip

\textbf{N.B.} Questo pattern sopperisce alla mancanza del doppio invio, \underline{double dispatch}  (overloading dinamico).
\medskip

Supponiamo di prendere in considerazione l'esempio dell'esercitazione Expression.

Quindi avremo la nostra interfaccia ExpressionVisitor con due metodi, visitConstant e visitSum che prendono in input, rispettivamente, un Constant ed un Sum e una 
sua implementazione concreta, ParenthesisRepresentationVisitor. 

Nell'interfaccia Expression avremo il metodo accept(...) che prende in input ExpressionVisitor ed infine avremo le classi Constant e Sum che implementeranno 
accept(...) chiamando rispettivamente visitConstant e visitSum.

\section{Double dispatch}

Supponiamo di avere il seguente test

\begin{lstlisting}
@Test
public void testReprImproved() {
    ExpressionVisitor r = new ParenthesisRepresentationVisitor();
    Expression exp = new Sum(
                        new Constant(10),
                        new Constant(5)
                        );
    assertEquals("(10 + 5)", exp.accept(r));
}
\end{lstlisting}

Quando chiamiamo exp.accept(r), exp, staticamente, è di tipo Expression ma a runtime è di tipo Sum e quindi, per il binding dinamico, viene chiamato 
Sum.accept(...) che chiama visitor.visitSum(...).

Visitor, staticamente, è di tipo ExpressionVisitor che, a runtime, è ParenthesisRepresentationVisitor, quindi per il binding dinamico, viene chiamato 
ParenthesisRepVisitor.visitSum.

Ecco perchè double dispatch, ovvero abbiamo usato due volte il meccanismo del binding dinamico, noi vogliamo eseguire un certo metodo in base al tipo effettivo 
(runtime) dell’elemento.

\section{Perchè si arriva al Visitor}

\subsection{Overloading statico}
Prendiamo sempre come esempio expression e definiamo la classe ParenthesisRepresentation che implementa il metodo repr in overloading, uno accetta Sum e l'altro 
Constant che ritorna un stringa.
\begin{lstlisting}
public class ParenthesisRepresentation {
  
  public String repr(Constant c) {
    return "Constant";
  }
  
  public String repr(Sum s) {
    return "Sum";
  }
}   
\end{lstlisting}

Supponiamo ora che vogliamo migliorare i metodi, il primo ritorna il valore della constante in stringa, mentre il secondo deve rappresentare l'operazione di 
somma di due valori.
\begin{lstlisting}
public class ParenthesisRepresentation {
  public String repr(Constant c) {
    return "" + c.getValue();
  }

  public String repr(Sum s) {
    return "(" +
            repr(s.getLeft()) +
            " + " +
            repr(s.getRight()) +
            ")";
  }
}
\end{lstlisting}

Il compilatore ci segnalerà due warning in repr(Sum s) perchè i metodi getLeft() e getRight() ritornano un Expression e noi non abbiamo un metodo repr che 
accetta un expression.

Quindi aggiungiamo il terzo metodo repr che lancia un eccezione perchè noi pensiamo che non venga mai chiamata
\begin{lstlisting}
public String repr(Expression expr) {
  throw new UnsupportedOperationException("Caso non contemplato");
}   
\end{lstlisting}

ed invece verrà chiamata perchè i i metodi getLeft() e getRight() ritornano un Expression e, siccome l'overloading è statico, viene chiamato repr(Expression expr).

\subsection{Simulare l’overloading dinamico con un dispatch manuale}

Prendiamo il metodo repr(Expression expr) e usiamo instanceof e cast
\begin{lstlisting}
public class ParenthesisRepresentation {
  public String repr(Expression expr) {
    if (expr instanceof Constant) {
      return repr((Constant) expr);
    } else if (expr instanceof Sum) {
      return repr((Sum) expr);
    }
    throw new UnsupportedOperationException("Caso non contemplato");
  }
  
  public String repr(Constant c) {...}
  public String repr(Sum s) {...}
}
\end{lstlisting}

tutto ok ma il tutto è macchinoso e poco leggibile, usiamo instanceof e cast (che è “error prone”), è inefficiente e c'è lo spauracchio dell'eccezione dietro 
l'angolo.

Infatti se passiamo Multipliction, avremo eccezione.

\subsection{dispatch map}

Quindi usiamo una Map per selezionare il metodo in base al tipo effettivo dell’oggetto dove la chiave è la classe mentre il valore valore è dato dalla lambda 
che chiama il metodo
\begin{lstlisting}
public class ParenthesisRepresentation {
  private Map<Class<? extends Expression>,Function<Expression, String>> dispatchMap = new HashMap<>();

  public ParenthesisRepresentation() {
    dispatchMap.put(Constant.class, e -> repr((Constant) e));
    dispatchMap.put(Sum.class, e -> repr((Sum) e));
  }

  public String repr(Expression expr) {
    Function<Expression, String> method = dispatchMap.get(expr.getClass());
    if (method == null)
      throw new UnsupportedOperationException("Caso non contemplato");
    return method.apply(expr);
  }
  public String repr(Constant c) {...}
  public String repr(Sum s) {...}
}
\end{lstlisting}

Molto difficile da implementare e mantenere, ci sono sempre i downcast che sono error-pronee c’è sempre la possibilità dell’UOException, e più facile compiere 
imprecisazioni o sviste di codice però è più efficiente del dispatch manuale.

\section{Collaborazioni}

\begin{figure}[H]
  \centering
  \includegraphics[width=8cm]{visitor/struttura_visitor}
\end{figure}

Un client crea un ConcreteVisitor e visita la struttura a oggetti.

Quando un elemento viene visitato chiama il metodo del visitor che corrisponde alla sua classe, passando se stesso come argomento, in questo modo il visitor può 
accedere allo stato dell’oggetto, se necessario.
\smallskip

Aggiungere una nuova classe alla gerarchia, come Multipliction, comporterà aggiungere un nuovo metodo visitXXX, in ExpressionVisitor, che prenderà in input 
Multipliction ed implentare accept(...) in Multipliction e l'aggiunta di questo metodo comporterà errori di compilazione nei visitor concreti, quindi occorrerà 
implementarlo.

\section{Varianti Visitor}

Supponiamo di voler rimuovere il metodo eval() da Expression e di implementare la valutazione di un espressione tramite un EvalVisitor, tale che 
EvalVisitor$<:$ExpressionVisitor.

\subsection{Visitor generico}

Dall'esempio precendete, noi sappiamo che in ExpressionVisitor i metodi visitXXX ritornano una stringa, ma a noi servirebbe che i metodi visitXXX di EvalVisitor 
ritornino un integer, quindi rendiamo ExpressionVisitor generico e, a sua volta, i metodi visitXXX e accept diventeranno generici.
\begin{multicols}{2}
\begin{lstlisting}
public interface ExpressionVisitor<T> {
  T visitConstant(Constant c);
  T visitSum(Sum s);
}
\end{lstlisting}
\columnbreak
\begin{lstlisting}
public interface Expression {
  <T> T accept(ExpressionVisitor<T> visitor);
}     
\end{lstlisting}
\end{multicols}

\begin{multicols}{2}
\begin{lstlisting}
public class Constant implements Expression {
  ...
  @Override
  public <T> T accept(ExpressionVisitor<T> visitor) {
    return visitor.visitConstant(this);
  }
}
\end{lstlisting}
\columnbreak
\begin{lstlisting}
public class Sum extends BinaryExpression {
  ...
  @Override
  public <T> T accept(ExpressionVisitor<T> visitor) {
    return visitor.visitSum(this);
  }
}
\end{lstlisting}
\end{multicols}
  
Andremo a specificare l'argomento di tipo solamente quando andremo ad implementare Expression, con ParenthesisRepresentationVisitor o EvalVisitor che saranno,
rispettivamente, String e Integer.
\begin{multicols}{2}
\begin{lstlisting}
public class ParenthesisRepresentationVisitor
            implements ExpressionVisitor<String> {
  public String repr(Constant c) {
    return "" + c.getValue();
  }

  public String repr(Sum s) {
    return "(" +
            repr(s.getLeft()) +
            " + " +
            repr(s.getRight()) +
            ")";
  }
}
\end{lstlisting}
\columnbreak
\begin{lstlisting}
public class EvalVisitor
           implements ExpressionVisitor<Integer> {
  @Override
  public Integer visitConstant(Constant c) {
    return c.getValue();
  }
  @Override
  public Integer visitSum(Sum s) {
    return s.getLeft().accept(this)
            +
            s.getRight().accept(this);
  }
}
\end{lstlisting}
\end{multicols}

\subsection{Visitor void}

I metodi visitXXX e accept sono void, ogni visitor concreto si mantiene uno stato, che viene usato per accumulare i risultati durante la visita e fornisce
un metodo per ottenere il risultato finale di tutta la vista.
\begin{multicols}{2}
\begin{lstlisting}
public class ParenthesisRepresentationVisitor 
                  implements ExpressionVisitor {
  private String representation = "";
  
  public String getRepresentation() {
    return representation;
  }

  @Override
  public void visitConstant(Constant c) {
    representation += c.getValue();
  }

  @Override
  public void visitSum(Sum e) {
    representation += "(";
    e.getLeft().accept(this);
    representation += " + ";
    e.getRight().accept(this);
    representation += ")";
  }
}
\end{lstlisting}
\columnbreak
\begin{lstlisting}
public class EvalVisitor 
          implements ExpressionVisitor {
  
  private int result = 0;
  public int getResult() {
    return result;
  }

  @Override
  public void visitConstant(Constant c) {
    result = c.getValue();
  }

  @Override
  public void visitSum(Sum e) {
    e.getLeft().accept(this);
    int temp = result;
    e.getRight().accept(this);
    result += temp;
  }
} 
\end{lstlisting}
\end{multicols}
\section{Coseguenze}

Dai vari tentativi pre-visitor ne emerge che non c’è modo di avere errori a runtime, se ci dimentichiamo di implementare un visit o un’accept avremo un errore 
dal compilatore.
\smallskip

Affinchè i visitor possano agire sugli elementi, l'interfaccia degli elementi deve fornire operazioni pubbliche di accesso e questo compromette l'incapsulamento.

E' facile aggiungere nuove operazioni, basta aggiungere un visitor concreto ma è difficile aggiungere un nuovo elemento da visitare perchè bisogna definire un nuovo 
metodo in visitor che poi dovrà essere implementato dalle classi concrete del visitor e il nuovo elemento da visitare deve implementare accept.

Per questo motivo, il visitor andrebbe usato quando la gerarchia degli oggetti è \underline{stabile}.

Una soluzione a questo, potrebbe essere quella di fornire un'implementazione dei metodi visitXXX in una classe astratta che implementa Visitor, cosicchè i client 
che implementeranno Visitor, estenderanno la classe astratta e ridefiniranno i metodi di loro interesse (questo però è un altro pattern).
\smallskip

Simulando il binding dinamico, facciamo a meno di instanceof e downcast, inoltre, se il linguaggio permettesse l'overloading dinamico, allora non ci sarebbe bisogno 
del metodo accept nelle classi da visitare ma comunque, quest'ultime, dovrebbero sempre fornire punti di accesso alla classe.