\section{Interfaccia}
Un'interfaccia è un meccanismo per definire un \underline{contratto} tra due parti, il fornitore del servizio (l'interfaccia stessa) e le classi che vogliono 
che i loro oggetti siano utilizzabili con quel servizio.
\smallskip

Prendiamo ad esempio l'interfaccia Comparable$<$T$>$, avente un metodo, compareTo(T o), che restiruisce un intero e assicura di confrontare solo oggetti dello 
stesso tipo.

Se una classe decidesse di implementare questa interfaccia, quindi fornire un'implementazione del metodo, allora i suoi oggetti potrebbero essere ordinati da 
java.

Il contratto è che x.compareTo(y) deve restituire:
\begin{itemize}
    \item un intero positivo se x viene dopo di y;
    \item un intero negativo nel caso contrario;
    \item 0 altrimenti.
\end{itemize}

\textbf{N.B.}La classe String implementa di suo questa interfaccia e implementa compareTo con il confronto lessicografico.

\medskip
Questo spezzone di codice funziona in quanto \textit{Arrays.sort} riesce a ordinare oggetti la cui classe implementa Comparable e, come detto prima, String la 
implementa.
\begin{lstlisting}
String[] friends = { "Peter", "Paul", "Mary" };
Arrays.sort(friends); // friends is now ["Mary", "Paul", "Peter"]
System.out.println(Arrays.toString(friends));
\end{lstlisting}

Così come potremmo ordinare oggetti \textit{Employee} in base al loro nome
\begin{lstlisting}
public class Employee implements Comparable<Employee> {
    private String name;

    @Override
    public int compareTo(Employee other) {
        return name.compareTo(other.getName());
    }
}
\end{lstlisting}
dove, in questo caso, il compareTo di Employee delega il confronto al compareTo di String.
\smallskip

Se volessimo ordinare Employee/String con un altro criterio, non potremmo farlo in quanto non sarebbe possibile definire due metodi compareTo e non sarebbe possibile
modificare la classe java però esiste una variante di Arrays.sort che, oltre ad accettare una lista da ordinare, accetta un'altra interfaccia, Comparator$<$T$>$, 
avente un metodo compare(T o1, T o2) che restituisce un intero.

Quindi per definire un nuovo criterio, dovremo definire una nuova classe, che implementa Comparator e passarla al metodo Arrays.sort.
\begin{lstlisting}
public class SortDemo {
    public static void main(String[] args) {
        String[] friends = new String[] { "Peter", "Paul", "Mary" };
        Arrays.sort(friends, new LengthComparator());
        // [Paul, Mary, Peter]
        System.out.println(Arrays.toString(friends));
    }
}
...
class LengthComparator implements Comparator<String> {
    public int compare(String first, String second) {
        return first.length() - second.length();
    }
}
\end{lstlisting}

Se il metodo di confronto ci servisse solo in quel punto di SortDemo, saremmo comunque costretti a definire una classe e istanziarla.

\subsection{Classe anonima}

Per questo motivo ci sono le \underline{classi anonime}, un meccanismo che riduce la verbosità del codice
\begin{itemize}
    \item che permettono di dichiarare e istanziare una classe allo stesso tempo;
    \item sono simili alle classi locali, solo che non hanno un nome;
    \item la loro invocazione avviene come quella di un costruttore, solo che al suo interno c'è una classe vera e propria.
\end{itemize}

\begin{lstlisting}
public class SortDemo {
    public static void main(String[] args) {
        String[] friends = new String[] { "Peter", "Paul", "Mary" };
        Arrays.sort(friends, new Comparator<String>() {
            public int compare(String first, String second) {
                return first.length() - second.length();
            }
        });
        // [Paul, Mary, Peter]
        System.out.println(Arrays.toString(friends));
    }
}
\end{lstlisting}

Fino ad ora, abbiamo visto due interfacce, avente un \underline{singolo metodo}, il contratto che stiamo usando \underline{dipende} dal quel singolo
metodo e se non ci fosse bisogno di mantenersi uno stato per implementare quel metodo, allora sarebbe più comodo poter specificare solo quel singolo blocco di codice 
invece di creare una classe che implementa l’interfaccia e la istanzia oppure creare una classe anonima.