\chapter{FACADE}

Pattern \underline{comportamentale} (descrive come classi e oggetti interagiscono e si distribuiscono le responsabilità) con l'intento di fornire un'interfaccia 
unificata ad un set di interfacce di un sottosistema complesso in modo tale da semplificarne l'utilizzo, nascondendo i dettagli interni del sottosistemadi 
interfacce e delegando alle stesse, rendendo il codice più pulito.

Il suo intento è quello di fornire vista semplificata di default, di alto livello disaccoppiando i client dai dettagli del sottosistema e facilitanto la 
strutturazione a strati dove uno strato comunica col sottostante attraverso il facade.

Usando i pattern e seguendo i principi realizziamo sistemi complessi costituiti da tanti piccoli componenti, disaccoppiati il più possibile e facilmente 
intercambiabili (riusabili), eventualmente anche a runtime.

Chi vuole, può manipolare e interagire direttamente coi singoli componenti tramite tipi astratti oppure può interagire con la “facciata” semplificata dei Facade.

\section{Esempio Visitor con Expression}

Per visitare un'espressione avevamo a disposizione due visitor, ExpressionTypeSystemVisitor ed ExpressionEvaluatorVisitor.

Tutti i client, per avere il tipo di un’espressione, dovevano creare un visitor e chiamare il metodo accept su un'Expression
\begin{lstlisting}
    Expression e = new Multiplication(new Constant(1), new Constant(2));
    Class<?> result = e.accept(new ExpressionTypeSystemVisitor());
\end{lstlisting}

Con il facade basta aggiungere un intermediario, ExpressioneTypeSystem ed ExpressionEvaluator, che attraverso un metodo, delegano al visitor appropriato
\begin{multicols}{2}
\begin{lstlisting}
    public class ExpressionTypeSystem {
        public Class<?> computeType(Expression e) {
            return e.accept
                (new ExpressionTypeSystemVisitor());
        }
    }
\end{lstlisting}
\columnbreak
\begin{lstlisting}
    Expression e = new Multiplication(new Constant(1), new Constant(2));
    Class<?> result = new ExpressionTypeSystem()
                        .computeType(e);
\end{lstlisting}
\end{multicols}

Stessa cosa con ExpressionEvaluator dove il metodo, in questo caso eval(), farà accept di un ExpressionEvaluatorVisitor su un'Expression.

Nell'esempio del visitor, il valutatore dava per scontato che le espressioni fossero già state controllate dal type system e quindi i client dovevano usare i due 
componenti nell’ordine giusto, prima il type system e poi, se non ci sono errori di tipo, il valutatore
\begin{lstlisting}
Expression e = new Multiplication(new Constant(1), new Constant(2));
Class<?> type = new ExpressionTypeSystem().computeType(e);
// in case catch exception
Object result = new ExpressionEvaluator().eval(e)
\end{lstlisting}

Quindi, a questo possiamo nascondere ulteriormente, ovverro aggiungere un nuovo facade del tipo
\begin{lstlisting}
public class ExpressionInterpreter {
    public Object interpret(Expression expression) throws ExpressionInterpreterException {
        try {
            // ignore type result, as long as it's well-typed
            new ExpressionTypeSystem().computeType(expression);
            return new ExpressionEvaluator().eval(expression);
        } catch (TypeSystemException e) {
            throw new ExpressionInterpreterException("Expression not well-typed", e);
        }
    }
}
\end{lstlisting}

cosicchè alla fine i Client dovranno solamente scrivere
\begin{lstlisting}
Expression e = new Multiplication(new Constant(1), new Constant(2));
// in case catch exception
Object result = new ExpressionInterpreter().interpret(e);
\end{lstlisting}

\textbf{N.B.} I Client possono comunque usare direttamente tutto quello che c'è dietro.

\section{Altro caso di utilizzo}

Supponiamo che il client sia il nostro codice ed il sottosistema siano librerie esterne.

Il Facade farà da tramite fra il nostro codice e la libreria esterna cosicchè se un giorno decidessimo di cambiare la librearia, dovremo solo modificare il Facade.