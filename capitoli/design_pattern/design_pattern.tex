\chapter{DESIGN PATTERN}

Sono una soluzione generica e riusabile per un problema comune, ci dicono come risolvere un problema in un certo contesto software, fornendo chiare linee guida.

Nel senso pratico sono schemi, modelli che descrivono relazioni tra interfacce e classi e interazioni tra oggetti e sono pensati per risolvere specifici problemi di design, 
rendendo il codice flessibile, elegante e riusabile. 

Ogni pattern è composto da 4 elementi, il \textit{nome} per riferirsi al pattern, il \textit{problema} che ci dice quando si applica e non 
(in alcuni casi ci sono delle condizioni da soddisfare), la \textit{soluzione} che descrive gli elementi (classi, interfacce, oggetti) e le loro relazioni 
(responsabilità e collaborazioni) e le \textit{conseguenze} che mostrano i risultati e i compromessi dall’applicazione del pattern.
\medskip

\textbf{N.B.} Alcuni pattern sopperiscono alle mancanze di funzionalità/caratteristiche del linguaggio di programmazione 
(ad esempio il Visitor sopperisce alla mancanza dell'overloading dinamico, decorato ed altri alla composizione di funzioni).
\medskip

Si differenziano per \underline{purpose} (proposito) che riguardano le tipologie di pattern (creazionali, strutturali e comportamentali) e per \underline{scope} 
(campo di azione) che riguardano le relazioni tra classi e sottoclassi, che sfruttano ereditarietà e overriding e oggetti che sfruttano object composition e delegation.
\smallskip

Nei pattern
\begin{itemize}
    \item creazionali i class patterns delegano a sottoclassi mentre gli object patterns delegano a un altro oggetto;
    \item strutturali i class patterns usano l’inheritance per comporre classi mentre gli object patterns descrivono modi per assemblare oggetti;
    \item comportamentali i class patterns usano l’inheritance per descrivere algoritmi e il “flow of control” mentre gli object patterns descrivono come un gruppo
    di oggetti cooperano per eseguire un certo task.
\end{itemize}.

\section{Notazione Uml}

Unified Modeling Language, una notazione formale per la specifica, costruzione, visualizzazione e documentazione del modello di un sistema software.

Utile sia per documentazione che nella fase di sviluppo.

\section{Interaction diagram}

Mostra l’ordine in cui le richieste fra oggetti vengono eseguite.