\subsection{Optional}
Un oggetto Optional$<$T$>$ è un wrapper di un oggetto T oppure nessun oggetto.

L'idea di base sull’utilizzo di un oggetto Optional è quella di usare i suoi metodi che permettono di produrre un’alternativa se il valore non è presente o consumarlo.

Questo oggetto deve essere usato in modo appropriato, altrimenti si hanno gli stessi problemi che si hanno con null, ad esempio
\begin{lstlisting}
Optional<T> optionalValue = ...;
optionalValue.get().someMethod();

T value = ...;
value.someMethod();
\end{lstlisting}

Questi due blocchi sono uguali, se optionalValue non contenesse alcun valore, allora avremo una NPE, stessa cosa per quest'altro esempio
\begin{lstlisting}
if (optionalValue.isPresent())
    optionalValue.get().someMethod();

if (value != null)
    value.someMethod();
\end{lstlisting}


